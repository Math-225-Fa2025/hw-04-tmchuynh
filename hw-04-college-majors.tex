% Options for packages loaded elsewhere
\PassOptionsToPackage{unicode}{hyperref}
\PassOptionsToPackage{hyphens}{url}
\documentclass[
]{article}
\usepackage{xcolor}
\usepackage[margin=1in]{geometry}
\usepackage{amsmath,amssymb}
\setcounter{secnumdepth}{5}
\usepackage{iftex}
\ifPDFTeX
  \usepackage[T1]{fontenc}
  \usepackage[utf8]{inputenc}
  \usepackage{textcomp} % provide euro and other symbols
\else % if luatex or xetex
  \usepackage{unicode-math} % this also loads fontspec
  \defaultfontfeatures{Scale=MatchLowercase}
  \defaultfontfeatures[\rmfamily]{Ligatures=TeX,Scale=1}
\fi
\usepackage{lmodern}
\ifPDFTeX\else
  % xetex/luatex font selection
\fi
% Use upquote if available, for straight quotes in verbatim environments
\IfFileExists{upquote.sty}{\usepackage{upquote}}{}
\IfFileExists{microtype.sty}{% use microtype if available
  \usepackage[]{microtype}
  \UseMicrotypeSet[protrusion]{basicmath} % disable protrusion for tt fonts
}{}
\makeatletter
\@ifundefined{KOMAClassName}{% if non-KOMA class
  \IfFileExists{parskip.sty}{%
    \usepackage{parskip}
  }{% else
    \setlength{\parindent}{0pt}
    \setlength{\parskip}{6pt plus 2pt minus 1pt}}
}{% if KOMA class
  \KOMAoptions{parskip=half}}
\makeatother
\usepackage{color}
\usepackage{fancyvrb}
\newcommand{\VerbBar}{|}
\newcommand{\VERB}{\Verb[commandchars=\\\{\}]}
\DefineVerbatimEnvironment{Highlighting}{Verbatim}{commandchars=\\\{\}}
% Add ',fontsize=\small' for more characters per line
\usepackage{framed}
\definecolor{shadecolor}{RGB}{248,248,248}
\newenvironment{Shaded}{\begin{snugshade}}{\end{snugshade}}
\newcommand{\AlertTok}[1]{\textcolor[rgb]{0.94,0.16,0.16}{#1}}
\newcommand{\AnnotationTok}[1]{\textcolor[rgb]{0.56,0.35,0.01}{\textbf{\textit{#1}}}}
\newcommand{\AttributeTok}[1]{\textcolor[rgb]{0.13,0.29,0.53}{#1}}
\newcommand{\BaseNTok}[1]{\textcolor[rgb]{0.00,0.00,0.81}{#1}}
\newcommand{\BuiltInTok}[1]{#1}
\newcommand{\CharTok}[1]{\textcolor[rgb]{0.31,0.60,0.02}{#1}}
\newcommand{\CommentTok}[1]{\textcolor[rgb]{0.56,0.35,0.01}{\textit{#1}}}
\newcommand{\CommentVarTok}[1]{\textcolor[rgb]{0.56,0.35,0.01}{\textbf{\textit{#1}}}}
\newcommand{\ConstantTok}[1]{\textcolor[rgb]{0.56,0.35,0.01}{#1}}
\newcommand{\ControlFlowTok}[1]{\textcolor[rgb]{0.13,0.29,0.53}{\textbf{#1}}}
\newcommand{\DataTypeTok}[1]{\textcolor[rgb]{0.13,0.29,0.53}{#1}}
\newcommand{\DecValTok}[1]{\textcolor[rgb]{0.00,0.00,0.81}{#1}}
\newcommand{\DocumentationTok}[1]{\textcolor[rgb]{0.56,0.35,0.01}{\textbf{\textit{#1}}}}
\newcommand{\ErrorTok}[1]{\textcolor[rgb]{0.64,0.00,0.00}{\textbf{#1}}}
\newcommand{\ExtensionTok}[1]{#1}
\newcommand{\FloatTok}[1]{\textcolor[rgb]{0.00,0.00,0.81}{#1}}
\newcommand{\FunctionTok}[1]{\textcolor[rgb]{0.13,0.29,0.53}{\textbf{#1}}}
\newcommand{\ImportTok}[1]{#1}
\newcommand{\InformationTok}[1]{\textcolor[rgb]{0.56,0.35,0.01}{\textbf{\textit{#1}}}}
\newcommand{\KeywordTok}[1]{\textcolor[rgb]{0.13,0.29,0.53}{\textbf{#1}}}
\newcommand{\NormalTok}[1]{#1}
\newcommand{\OperatorTok}[1]{\textcolor[rgb]{0.81,0.36,0.00}{\textbf{#1}}}
\newcommand{\OtherTok}[1]{\textcolor[rgb]{0.56,0.35,0.01}{#1}}
\newcommand{\PreprocessorTok}[1]{\textcolor[rgb]{0.56,0.35,0.01}{\textit{#1}}}
\newcommand{\RegionMarkerTok}[1]{#1}
\newcommand{\SpecialCharTok}[1]{\textcolor[rgb]{0.81,0.36,0.00}{\textbf{#1}}}
\newcommand{\SpecialStringTok}[1]{\textcolor[rgb]{0.31,0.60,0.02}{#1}}
\newcommand{\StringTok}[1]{\textcolor[rgb]{0.31,0.60,0.02}{#1}}
\newcommand{\VariableTok}[1]{\textcolor[rgb]{0.00,0.00,0.00}{#1}}
\newcommand{\VerbatimStringTok}[1]{\textcolor[rgb]{0.31,0.60,0.02}{#1}}
\newcommand{\WarningTok}[1]{\textcolor[rgb]{0.56,0.35,0.01}{\textbf{\textit{#1}}}}
\usepackage{longtable,booktabs,array}
\usepackage{calc} % for calculating minipage widths
% Correct order of tables after \paragraph or \subparagraph
\usepackage{etoolbox}
\makeatletter
\patchcmd\longtable{\par}{\if@noskipsec\mbox{}\fi\par}{}{}
\makeatother
% Allow footnotes in longtable head/foot
\IfFileExists{footnotehyper.sty}{\usepackage{footnotehyper}}{\usepackage{footnote}}
\makesavenoteenv{longtable}
\usepackage{graphicx}
\makeatletter
\newsavebox\pandoc@box
\newcommand*\pandocbounded[1]{% scales image to fit in text height/width
  \sbox\pandoc@box{#1}%
  \Gscale@div\@tempa{\textheight}{\dimexpr\ht\pandoc@box+\dp\pandoc@box\relax}%
  \Gscale@div\@tempb{\linewidth}{\wd\pandoc@box}%
  \ifdim\@tempb\p@<\@tempa\p@\let\@tempa\@tempb\fi% select the smaller of both
  \ifdim\@tempa\p@<\p@\scalebox{\@tempa}{\usebox\pandoc@box}%
  \else\usebox{\pandoc@box}%
  \fi%
}
% Set default figure placement to htbp
\def\fps@figure{htbp}
\makeatother
\setlength{\emergencystretch}{3em} % prevent overfull lines
\providecommand{\tightlist}{%
  \setlength{\itemsep}{0pt}\setlength{\parskip}{0pt}}
\usepackage[]{natbib}
\bibliographystyle{plainnat}
\usepackage{bookmark}
\IfFileExists{xurl.sty}{\usepackage{xurl}}{} % add URL line breaks if available
\urlstyle{same}
\hypersetup{
  pdftitle={HW 04 - What should I major in?},
  pdfauthor={Your Name},
  hidelinks,
  pdfcreator={LaTeX via pandoc}}

\title{HW 04 - What should I major in?}
\author{Your Name}
\date{}

\begin{document}
\maketitle

{
\setcounter{tocdepth}{2}
\tableofcontents
}
\begin{figure}
\includegraphics[width=0.8\linewidth]{img/graduate} \caption{Photo by Marleena Garris on Unsplash}\label{fig:photo}
\end{figure}

The first step in the process of turning information into knowledge
process is to summarize and describe the raw information - the data. In
this assignment we explore data on college majors and earnings,
specifically the data begin the FiveThirtyEight story
\href{https://fivethirtyeight.com/features/the-economic-guide-to-picking-a-college-major/}{``The
Economic Guide To Picking A College Major''}.

These data originally come from the American Community Survey (ACS)
2010-2012 Public Use Microdata Series. While this is outside the scope
of this assignment, if you are curious about how raw data from the ACS
were cleaned and prepared, see
\href{https://github.com/fivethirtyeight/data/blob/master/college-majors/college-majors-rscript.R}{the
code} FiveThirtyEight authors used.

We should also note that there are many considerations that go into
picking a major. Earnings potential and employment prospects are two of
them, and they are important, but they don't tell the whole story. Keep
this in mind as you analyze the data.

\section{Getting started}\label{getting-started}

Go to the course GitHub organization and locate your homework repo,
which should be named \texttt{hw-04-YOUR\_GITHUB\_USERNAME}. Grab the
URL of the repo, and clone it in RStudio. First, open the R Markdown
document \texttt{hw-04.Rmd} and Knit it. Make sure it compiles without
errors. The output will be in the file markdown \texttt{.md} file with
the same name.

\subsection{Warm up}\label{warm-up}

Before we introduce the data, let's warm up with some simple exercises.

\begin{itemize}
\tightlist
\item
  Update the YAML, changing the author name to your name, and
  \textbf{knit} the document.
\item
  Commit your changes with a meaningful commit message.
\item
  Push your changes to GitHub.
\item
  Go to your repo on GitHub and confirm that your changes are visible in
  your Rmd \textbf{and} md files. If anything is missing, commit and
  push again.
\end{itemize}

\subsection{Packages}\label{packages}

We'll use the \textbf{tidyverse} package for much of the data wrangling
and visualisation, the \textbf{scales} package for better formatting of
labels on visualisations, and the data lives in the
\textbf{fivethirtyeight} package. These packages are already installed
for you. You can load them by running the following in your Console:

\begin{Shaded}
\begin{Highlighting}[]
\FunctionTok{library}\NormalTok{(tidyverse)}
\FunctionTok{library}\NormalTok{(scales)}
\FunctionTok{library}\NormalTok{(fivethirtyeight)}
\end{Highlighting}
\end{Shaded}

\subsection{Data}\label{data}

The data can be found in the \textbf{fivethirtyeight} package, and it's
called \texttt{college\_recent\_grads}. Since the dataset is distributed
with the package, we don't need to load it separately; it becomes
available to us when we load the package. You can find out more about
the dataset by inspecting its documentation, which you can access by
running \texttt{?college\_recent\_grads} in the Console or using the
Help menu in RStudio to search for \texttt{college\_recent\_grads}. You
can also find this information
\href{https://fivethirtyeight-r.netlify.app/reference/college_recent_grads.html}{here}.

You can also take a quick peek at your data frame and view its
dimensions with the \texttt{glimpse} function.

\begin{Shaded}
\begin{Highlighting}[]
\FunctionTok{glimpse}\NormalTok{(college\_recent\_grads)}
\end{Highlighting}
\end{Shaded}

The \texttt{college\_recent\_grads} data frame is a trove of
information. Let's think about some questions we might want to answer
with these data:

\begin{itemize}
\tightlist
\item
  Which major has the lowest unemployment rate?
\item
  Which major has the highest percentage of women?
\item
  How do the distributions of median income compare across major
  categories?
\item
  Do women tend to choose majors with lower or higher earnings?
\end{itemize}

In the next section we aim to answer these questions.

\section{Exercises}\label{exercises}

\subsection{Which major has the lowest unemployment
rate?}\label{which-major-has-the-lowest-unemployment-rate}

In order to answer this question all we need to do is sort the data. We
use the \texttt{arrange} function to do this, and sort it by the
\texttt{unemployment\_rate} variable. By default \texttt{arrange} sorts
in ascending order, which is what we want here -- we're interested in
the major with the \emph{lowest} unemployment rate.

\begin{Shaded}
\begin{Highlighting}[]
\NormalTok{college\_recent\_grads }\SpecialCharTok{\%\textgreater{}\%}
  \FunctionTok{arrange}\NormalTok{(unemployment\_rate)}
\end{Highlighting}
\end{Shaded}

This gives us what we wanted, but not in an ideal form. First, the name
of the major barely fits on the page. Second, some of the variables are
not that useful (e.g.~\texttt{major\_code}, \texttt{major\_category})
and some we might want front and center are not easily viewed
(e.g.~\texttt{unemployment\_rate}).

We can use the \texttt{select} function to choose which variables to
display, and in which order:

\begin{marginfigure}
Note how easily we expanded our code with adding another step to our
pipeline, with the pipe operator: \texttt{\%\textgreater{}\%}.
\end{marginfigure}

\begin{Shaded}
\begin{Highlighting}[]
\NormalTok{college\_recent\_grads }\SpecialCharTok{\%\textgreater{}\%}
  \FunctionTok{arrange}\NormalTok{(unemployment\_rate) }\SpecialCharTok{\%\textgreater{}\%}
  \FunctionTok{select}\NormalTok{(rank, major, unemployment\_rate)}
\end{Highlighting}
\end{Shaded}

Ok, this is looking better, but do we really need to display all those
decimal places in the unemployment variable? Not really!

We can use the \texttt{percent()} function to clean up the display a
bit.

\begin{Shaded}
\begin{Highlighting}[]
\NormalTok{college\_recent\_grads }\SpecialCharTok{\%\textgreater{}\%}
  \FunctionTok{arrange}\NormalTok{(unemployment\_rate) }\SpecialCharTok{\%\textgreater{}\%}
  \FunctionTok{select}\NormalTok{(rank, major, unemployment\_rate) }\SpecialCharTok{\%\textgreater{}\%}
  \FunctionTok{mutate}\NormalTok{(}\AttributeTok{unemployment\_rate =} \FunctionTok{percent}\NormalTok{(unemployment\_rate))}
\end{Highlighting}
\end{Shaded}

\subsection{Which major has the highest percentage of
women?}\label{which-major-has-the-highest-percentage-of-women}

To answer such a question we need to arrange the data in descending
order. For example, if earlier we were interested in the major with the
highest unemployment rate, we would use the following:

\begin{marginfigure}
The \texttt{desc} function specifies that we want
\texttt{unemployment\_rate} in descending order.
\end{marginfigure}

\begin{Shaded}
\begin{Highlighting}[]
\NormalTok{college\_recent\_grads }\SpecialCharTok{\%\textgreater{}\%}
  \FunctionTok{arrange}\NormalTok{(}\FunctionTok{desc}\NormalTok{(unemployment\_rate)) }\SpecialCharTok{\%\textgreater{}\%}
  \FunctionTok{select}\NormalTok{(rank, major, unemployment\_rate)}
\end{Highlighting}
\end{Shaded}

\begin{enumerate}
\def\labelenumi{\arabic{enumi}.}
\tightlist
\item
  Using what you've learned so far, arrange the data in descending order
  with respect to proportion of women in a major, and display only the
  major, the total number of people with major, and proportion of women.
  Show only the top 3 majors by adding \texttt{top\_n(3)} at the end of
  the pipeline.
\end{enumerate}

\subsection{How do the distributions of median income compare across
major
categories?}\label{how-do-the-distributions-of-median-income-compare-across-major-categories}

\begin{marginfigure}
A percentile is a measure used in statistics indicating the value below
which a given percentage of observations in a group of observations
fall. For example, the 20th percentile is the value below which 20\% of
the observations may be found. (Source:
\href{https://en.wikipedia.org/wiki/Percentile}{Wikipedia})
\end{marginfigure}

There are three types of incomes reported in this data frame:
\texttt{p25th}, \texttt{median}, and \texttt{p75th}. These correspond to
the 25th, 50th, and 75th percentiles of the income distribution of
sampled individuals for a given major.

\begin{enumerate}
\def\labelenumi{\arabic{enumi}.}
\setcounter{enumi}{1}
\tightlist
\item
  Why do we often choose the median, rather than the mean, to describe
  the typical income of a group of people?
\end{enumerate}

The question we want to answer ``How do the distributions of median
income compare across major categories?''. We need to do a few things to
answer this question: First, we need to group the data by
\texttt{major\_category}. Then, we need a way to summarize the
distributions of median income within these groups. This decision will
depend on the shapes of these distributions. So first, we need to
visualize the data.

We use the \texttt{ggplot()} function to do this. The first argument is
the data frame, and the next argument gives the mapping of the variables
of the data to the \texttt{aes}thetic elements of the plot.

Let's start simple and take a look at the distribution of all median
incomes, without considering the major categories.

\begin{Shaded}
\begin{Highlighting}[]
\FunctionTok{ggplot}\NormalTok{(}\AttributeTok{data =}\NormalTok{ college\_recent\_grads, }\AttributeTok{mapping =} \FunctionTok{aes}\NormalTok{(}\AttributeTok{x =}\NormalTok{ median)) }\SpecialCharTok{+}
  \FunctionTok{geom\_histogram}\NormalTok{()}
\end{Highlighting}
\end{Shaded}

Along with the plot, we get a message:

\begin{verbatim}
`stat_bin()` using `bins = 30`. Pick better value with `binwidth`.
\end{verbatim}

This is telling us that we might want to reconsider the binwidth we
chose for our histogram -- or more accurately, the binwidth we didn't
specify. It's good practice to always think in the context of the data
and try out a few binwidths before settling on a binwidth. You might ask
yourself: ``What would be a meaningful difference in median incomes?''
\$1 is obviously too little, \$10000 might be too high.

\begin{enumerate}
\def\labelenumi{\arabic{enumi}.}
\setcounter{enumi}{2}
\tightlist
\item
  Try binwidths of \$1000 and \$5000 and choose one. Explain your
  reasoning for your choice. Note that the binwidth is an argument for
  the \texttt{geom\_histogram} function. So to specify a binwidth of
  \$1000, you would use \texttt{geom\_histogram(binwidth\ =\ 1000)}.
\end{enumerate}

We can also calculate summary statistics for this distribution using the
\texttt{summarise} function:

\begin{Shaded}
\begin{Highlighting}[]
\NormalTok{college\_recent\_grads }\SpecialCharTok{\%\textgreater{}\%}
  \FunctionTok{summarise}\NormalTok{(}\AttributeTok{min =} \FunctionTok{min}\NormalTok{(median), }\AttributeTok{max =} \FunctionTok{max}\NormalTok{(median),}
            \AttributeTok{mean =} \FunctionTok{mean}\NormalTok{(median), }\AttributeTok{med =} \FunctionTok{median}\NormalTok{(median),}
            \AttributeTok{sd =} \FunctionTok{sd}\NormalTok{(median), }
            \AttributeTok{q1 =} \FunctionTok{quantile}\NormalTok{(median, }\AttributeTok{probs =} \FloatTok{0.25}\NormalTok{),}
            \AttributeTok{q3 =} \FunctionTok{quantile}\NormalTok{(median, }\AttributeTok{probs =} \FloatTok{0.75}\NormalTok{))}
\end{Highlighting}
\end{Shaded}

\begin{enumerate}
\def\labelenumi{\arabic{enumi}.}
\setcounter{enumi}{3}
\item
  Based on the shape of the histogram you created in the previous
  exercise, determine which of these summary statistics is useful for
  describing the distribution. Write up your description (remember
  shape, center, spread, any unusual observations) and include the
  summary statistic output as well.
\item
  Plot the distribution of \texttt{median} income using a histogram,
  faceted by \texttt{major\_category}. Use the \texttt{binwidth} you
  chose in the earlier exercise.
\end{enumerate}

Now that we've seen the shapes of the distributions of median incomes
for each major category, we should have a better idea for which summary
statistic to use to quantify the typical median income.

\begin{enumerate}
\def\labelenumi{\arabic{enumi}.}
\setcounter{enumi}{5}
\tightlist
\item
  Which major category has the highest typical (you'll need to decide
  what this means) median income? Use the partial code below, filling it
  in with the appropriate statistic and function. Also note that we are
  looking for the highest statistic, so make sure to arrange in the
  correct direction.
\end{enumerate}

\begin{Shaded}
\begin{Highlighting}[]
\NormalTok{college\_recent\_grads }\SpecialCharTok{\%\textgreater{}\%}
  \FunctionTok{group\_by}\NormalTok{(major\_category) }\SpecialCharTok{\%\textgreater{}\%}
  \FunctionTok{summarise}\NormalTok{(}\AttributeTok{\_\_\_ =} \FunctionTok{\_\_\_}\NormalTok{(median)) }\SpecialCharTok{\%\textgreater{}\%}
  \FunctionTok{arrange}\NormalTok{(\_\_\_)}
\end{Highlighting}
\end{Shaded}

\begin{enumerate}
\def\labelenumi{\arabic{enumi}.}
\setcounter{enumi}{6}
\tightlist
\item
  Which major category is the least popular in this sample? To answer
  this question we use a new function called \texttt{count}, which first
  groups the data and then counts the number of observations in each
  category (see below). Add to the pipeline appropriately to arrange the
  results so that the major with the lowest observations is on top.
\end{enumerate}

\begin{Shaded}
\begin{Highlighting}[]
\NormalTok{college\_recent\_grads }\SpecialCharTok{\%\textgreater{}\%}
  \FunctionTok{count}\NormalTok{(major\_category)}
\end{Highlighting}
\end{Shaded}

🧶 ✅ ⬆️ Knit, \emph{commit, and push your changes to GitHub with an
appropriate commit message. Make sure to commit and push all changed
files so that your Git pane is cleared up afterwards.}

\subsection{All STEM fields aren't the
same}\label{all-stem-fields-arent-the-same}

One of the sections of the
\href{https://fivethirtyeight.com/features/the-economic-guide-to-picking-a-college-major/}{FiveThirtyEight
story} is ``All STEM fields aren't the same''. Let's see if this is
true.

First, let's create a new vector called \texttt{stem\_categories} that
lists the major categories that are considered STEM fields.

\begin{Shaded}
\begin{Highlighting}[]
\NormalTok{stem\_categories }\OtherTok{\textless{}{-}} \FunctionTok{c}\NormalTok{(}\StringTok{"Biology \& Life Science"}\NormalTok{,}
                     \StringTok{"Computers \& Mathematics"}\NormalTok{,}
                     \StringTok{"Engineering"}\NormalTok{,}
                     \StringTok{"Physical Sciences"}\NormalTok{)}
\end{Highlighting}
\end{Shaded}

Then, we can use this to create a new variable in our data frame
indicating whether a major is STEM or not.

\begin{Shaded}
\begin{Highlighting}[]
\NormalTok{college\_recent\_grads }\OtherTok{\textless{}{-}}\NormalTok{ college\_recent\_grads }\SpecialCharTok{\%\textgreater{}\%}
  \FunctionTok{mutate}\NormalTok{(}\AttributeTok{major\_type =} \FunctionTok{ifelse}\NormalTok{(major\_category }\SpecialCharTok{\%in\%}\NormalTok{ stem\_categories, }\StringTok{"stem"}\NormalTok{, }\StringTok{"not stem"}\NormalTok{))}
\end{Highlighting}
\end{Shaded}

Let's unpack this: with \texttt{mutate} we create a new variable called
\texttt{major\_type}, which is defined as \texttt{"stem"} if the
\texttt{major\_category} is in the vector called
\texttt{stem\_categories} we created earlier, and as
\texttt{"not\ stem"} otherwise.

\texttt{\%in\%} is a \textbf{logical operator}. Other logical operators
that are commonly used are

\begin{longtable}[]{@{}ll@{}}
\toprule\noalign{}
Operator & Operation \\
\midrule\noalign{}
\endhead
\bottomrule\noalign{}
\endlastfoot
\texttt{x\ \textless{}\ y} & less than \\
\texttt{x\ \textgreater{}\ y} & greater than \\
\texttt{x\ \textless{}=\ y} & less than or equal to \\
\texttt{x\ \textgreater{}=\ y} & greater than or equal to \\
\texttt{x\ !=\ y} & not equal to \\
\texttt{x\ ==\ y} & equal to \\
\texttt{x\ \%in\%\ y} & contains \\
x \textbar{} y & or \\
\texttt{x\ \&\ y} & and \\
\texttt{!x} & not \\
\end{longtable}

We can use the logical operators to also \texttt{filter} our data for
STEM majors whose median earnings is less than median for all majors'
median earnings, which we found to be \$36,000 earlier.

\begin{Shaded}
\begin{Highlighting}[]
\NormalTok{college\_recent\_grads }\SpecialCharTok{\%\textgreater{}\%}
  \FunctionTok{filter}\NormalTok{(}
\NormalTok{    major\_type }\SpecialCharTok{==} \StringTok{"stem"}\NormalTok{,}
\NormalTok{    median }\SpecialCharTok{\textless{}} \DecValTok{36000}
\NormalTok{  )}
\end{Highlighting}
\end{Shaded}

\begin{enumerate}
\def\labelenumi{\arabic{enumi}.}
\setcounter{enumi}{7}
\tightlist
\item
  Which STEM majors have median salaries equal to or less than the
  median for all majors' median earnings? Your output should only show
  the major name and median, 25th percentile, and 75th percentile
  earning for that major as and should be sorted such that the major
  with the highest median earning is on top.
\end{enumerate}

🧶 ✅ ⬆️ Knit, \emph{commit, and push your changes to GitHub with an
appropriate commit message. Make sure to commit and push all changed
files so that your Git pane is cleared up afterwards.}

\subsection{What types of majors do women tend to major
in?}\label{what-types-of-majors-do-women-tend-to-major-in}

\begin{enumerate}
\def\labelenumi{\arabic{enumi}.}
\setcounter{enumi}{8}
\tightlist
\item
  Create a scatterplot of median income vs.~proportion of women in that
  major, coloured by whether the major is in a STEM field or not.
  Describe the association between these three variables.
\end{enumerate}

\subsection{Further exploration}\label{further-exploration}

\begin{enumerate}
\def\labelenumi{\arabic{enumi}.}
\setcounter{enumi}{9}
\tightlist
\item
  Ask a question of interest to you, and answer it using summary
  statistic(s) and/or visualization(s).
\end{enumerate}

🧶 ✅ ⬆️ Knit, \emph{commit, and push your changes to GitHub with an
appropriate commit message. Make sure to commit and push all changed
files so that your Git pane is cleared up afterwards and review the md
document on GitHub to make sure you're happy with the final state of
your work.}

Now go back through your write up to make sure you've answered all
questions and all of your R chunks are properly labeled. Once you decide
that you are done with the homework, choose the knit drop down and
select \texttt{Knit\ to\ tufte\_handout} to generate a pdf. Download and
submit that pdf to Canvas.

\end{document}
